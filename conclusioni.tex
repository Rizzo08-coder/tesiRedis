\chapter{Conclusioni}
Dopo aver fornito una panoramica delle basi di dati NoSQL,
si è illustrata una modellazione astratta ed indipendente dal dbms specifico, grazie alla quale si ha una libertà
di progettazione dal sistema utilizzato, ad eccezione della fase di implementazione vera e propria.\\
Dal terzo capitolo si è preso in considerazione come caso di studio \textbf{Redis}, sono state fatte notare le caratteristiche peculiari e possibili ambiti di utilizzo.\\
Si è visto che non vengono implementate tutte le proprietà ACID; inoltre, certe proprietà, come la persistenza,
possono essere modificate a livello di file di configurazione del server.\\
Nel corso del quarto capitolo è stato fatto un confronto tra il Data Manipulation Language di SQL
e quello di Redis;
per Redis si è considerata la struttura dati Hash per questo paragone.\\
Nell'ultimo capitolo, l'applicativo software è stato progettato sfruttando due strutture dati differenti: String e RedisTimeSeries.\\
Per la visualizzazione del grafico è stata utilizzata la libreria JFreeChart.\\
Un possibile miglioramento potrebbe essere l'aumento della frequenza di campionamento da parte dei sensori, con un conseguente aumento di velocità di aggiornamento del grafico.
Per farlo con risultati estetici apprezzabili, si avrebbe la necessità
di implementare una libreria diversa da quella utilizzata, adatta ad intervalli di refresh piú brevi.\\
Inoltre, potremmo voler disporre di maggiori funzionalità e interattività associate alla visualizzazione del grafico, ad esempio per fare un confronto tra diversi sensori appartenente ad una stessa rete, oppure
ripercorrere indietro nel tempo i prelevamenti.
A questo punto, converrebbe utilizzare librerie che mettono a disposizione costrutti per ottenere grafici che dispongono di funzionalità piú avanzate.\\
Si potrebbe pensare di memorizzare i dati meno recenti in database non in memory, in modo da avere uno storico molto piú prolungato di dati, con garanzie di persistenza.
Redis verrebbe sfruttato solo per dati istantanei, o comunque per dati generati in un intervallo temporale abbastanza vicino a quello attuale.\\


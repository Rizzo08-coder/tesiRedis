\chapter{Conclusioni}
Dopo aver fornito una panoramica delle basi di dati NoSQL nel corso del primo capitolo,
è stata definita una modellazione astratta ed indipendente dal dbms specifico.
Dal terzo capitolo si è preso in considerazione come caso di studio \textbf{Redis}, sono state fatte notare le proprie caratteristiche
ed ambiti di utilizzo. Si è notato che non vengono implementate tutte le proprietà ACID, inoltre certe proprietà
possono essere modificate a livello di configurazione del server, oppure devono essere gestite a livello applicativo.
Nel corso del quarto capitolo è stato fatto un confronto tra il Data Manipulation Language di SQL
e quello di Redis, utilizzando come caso di studio un database che memorizza dati utente, in Redis è stata sfruttata la struttura dati Hash.
Nell'ultimo capitolo, l'applicativo software è stato progettato sfruttando due strutture dati differenti:String e RedisTimeSeries.
Per il disegno del grafico in Java è stata utilizzata la libreria JFreeChart.\\
Un possibile miglioramento potrebbe essere andare l'aumento della frequenza di campionamento del sensore; per fare questo, si avrebbe la necessità
di andare ad implementare una libreria diversa da JFreeChart, adatta ad una frequenza di aggiornamento maggiore.\\
Inoltre, potremmo pensare di avere un grafico che implementa maggiori funzionalità, ad esempio per fare un confronto tra diversi sensori all' interno di una rete, oppure
andare a ripercorrere indietro nel tempo i prelevamenti in caso di maggiore persistenza dei dati, magari potremmo pensare di utilizzare un database non in memory per fare questo,
quindi l'applicativo dovrebbe disporre di un database in memory per i dati istantanei ed uno differente per dati vecchi.

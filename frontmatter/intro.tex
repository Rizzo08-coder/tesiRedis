\chapter*{Introduzione}
Per decenni l'approccio alle basi di dati é sempre stato quello relazionale: la gestione dei dati e il relativo
linguaggio di riferimento per le interrogazioni sono stati punti inamovibili dell'IT.
Il modello relazionale é collaudato, conosciuto e adatto alle applicazioni piú esigenti.
La sua affidabilità è un elemento essenziale per il mondo aziendale e le applicazioni business in cui viene adottato.\\
In altri ambiti, però, specialmente nello sviluppo di applicazioni aperte al web, la grande affidabilità é stata
messa in secondo piano dal prezzo che si paga per averla. Quando si pensa ad applicazioni che possono essere
usate da milioni di utenti via Internet e con profili di carico a volte imprevedibili, scalabilità e performance
diventano piú importanti della correttezza delle informazioni.
Proprio da questa esigenza nasce il mondo oggi conosciuto come NoSQL.\\
A tal proposito, questa relazione vuole mostrare una panoramica delle basi di dati non relazionali NoSQL,
mostrando i vantaggi e i limiti di questo approccio. Inoltre, verrá riportata
una tassonomia in base al modello che viene utilizzato per la memorizzazione dei dati.
Ogni categoria porterà come conseguenza un diverso interfacciamento
con la base di dati stessa.\\
Nel secondo capitolo verrà presentato un modello di progettazione concettuale/logica
astratto ed adattabile alle differenti famiglie di NoSQL, completamente indipendente dalle implementazioni concrete.\\
Dal terzo capitolo verrà approfondito un particolare dbms chiave-valore: \textbf{Redis}.\\
A riguardo, verranno mostrate le principali peculiarità che lo contraddistinguono ed i propri punti di forza.
Verranno messe in evidenza le somiglianze e differenze che si hanno con le basi di dati tradizionali.\\
Nell'ultimo capitolo verranno messi in pratica gli studi fatti nei capitoli precedenti, si sfrutteranno
le caratteristiche di Redis per creare un applicativo utile in un ambito che si sta espandendo sempre di piú, ed in maniera sempre piú rapida
negli ultimi anni: l'Industrial of Thing.

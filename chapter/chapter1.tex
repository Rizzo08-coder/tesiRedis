\chapter{DataBase NoSQL}
I database \emph{NoSQL}, che sta per \emph{not only SQL}, sono database non tabellari che archiviano i dati
in maniera completamente differente dai classici relazionali.
Le caratteristiche principali sono la progettazione specifica per carichi elevati e il supporto nativo per la scalabilitá
orizzontale, la tolleranza agli errori e la memorizzazione dei dati in modo denormalizzato.
Infatti ogni elemento viene archiviato singolarmente con una chiave univoca, e la coerenza dei dati non viene garantita.
Questa impostazione fornisce un approccio molto piú flessibile alla memorizzazione dei dati rispetto a un database
relazionale, un controllo migliore e una maggiore semplicitá nelle applicazioni.

\section{Perche é nato \emph{NoSQL}?}
A partire dagli anni 2000 si é passati da un modello in cui le persone principali dell'IT erano sistemisti ad un modello
in cui le persone principali sono diventate gli sviluppatori. Tale passaggio ha comportato la nascita di database NoSQL
che sono fortemente orientati agli sviluppatori ed allo sviluppo Agile.
Inoltre i dati si sono trasformati passando dai classici strutturati a dati non strutturati (di differenti dimensioni,
semistrutturati, polimorfici..) che non permettevano di definire un modello relazionale organico e cosí i database NoSQL sono
diventati estremamente popolari perché permettono di lavorare principalmente con dati non strutturati anche di enormi
dimensioni.

\section{Perché dovrei utilizzare un database \emph{NoSQL}?}
I database NoSQL sono una soluzione ideale per molte applicazioni moderne, quali dispositivi mobili, Web e videogiochi
che richiedono strutture dati flessibili, scalabili, con prestazioni elevate ed altamente funzionali.
\begin{itemize}
    \item \texttt{Flessibilitá}: vengono offerti schemi flessibile che consentono uno sviluppo piú veloce. Quindi é una soluzione ideale
    per i dati semi-strutturati e non strutturati. É possibile arricchire le applicazioni di nuovi dati e informazioni
    senza dover sottostare ad una rigida struttura dei dati;
    \item \texttt{Scalabilitá}: grazie alla semplicitá vi é la possibilitá di \emph{scalare in orizzontale} in maniera estremamente
    efficiente. Infatti, si predilige l'utilizzo di cluster con molti nodi distribuiti, rispetto all'utilizzo di server centralizzati.
    Inoltre, vi é la possibilitá di aggiungere nodi a caldo in maniera completamente trasparente per l'utente finale;
    \item \texttt{Elevate Prestazioni}: grazie alla mancanza di operazioni di aggregazione dei dati("join") ed anche grazie
    all'introduzione di semplificazioni, come il mancato supporto delle transazioni ACID, si ha una elevata velocitá computazionale.
    \item \texttt{Altamente funzionali}: non vi é piú un linguaggio generale (SQL) come nei database relazionali, ma vi sono
    API in base al database specifico che si va ad utilizzare.
\end{itemize}


\section{Tipologie di \emph{NoSQL}}
Tipologie principali di database \emph{NoSQL}:
\begin{itemize}
    \item \texttt{documentali}
    \item \texttt{key-value}
    \item \texttt{colonnari}
    \item \texttt{a grafo}
\end{itemize}

\chapter{Interrogazioni Redis}
Le capacitá piú essenziali per il lavoro su database sono saperli popolare di informazioni in modo coerente e interrogare in modo
intelligente, per ricavare l'informazione che si sta cercando in mezzo a volumi di dati che possono essere immensi.
Quindi, per raggiungere questo obiettivo viene adoperato un linguaggio per l'interrogazione dei database, il quale serve per formulare delle
query, in modo adeguato alla struttura delle informazioni.
Nei database di tipo relazionale é presente il linguaggio SQL (Structured Query Language), il quale é comune a tutti i dbms di questo tipo, quindi
é diventato negli anni un vero e proprio standard.

\paragraph{nei NoSQL vi é un linguaggio standard da adoperare?\\}
i NoSQL, di conseguenza anche Redis, si caratterizzano da schemi non fissi, che possono variare in modo dinamico e, soprattutto, non vi sono vincoli referenziali che possono associare
diverse tabelle all'interno della base di dati (non vi sono operazioni di "join").
Per memorizzare e ricavare dati, i quali possono essere strutturati, semi-strutturati, non strutturati e polimorfici, un dbms noSQL utilizza un'ampia gamma di
tecnologie proprietarie in grado di fare questo.
Quindi, é compito del produttore del dbms di mettere a disposizione un insieme di comandi che permettano di sfruttare pienamente le potenzialità del NoSQL da lui fornito.
Ne deriva che non vi é un linguaggio standard comune ai NoSQL, o comunque comune ad una specifica categoria;
sono disponibili delle interfacce di comunicazione per i principali linguaggi di programmazione che vengono
offerte direttamente al programmatore, con le quali si riesce a gestire il dialogo con il dbms.

Sulla documentazione di Redis é presente un elenco completo di tutte le API disponibili per i principali linguaggi, tramite le quali si riesce a comunicare con il server in questione.

Il modo piú rapido e semplice per interagire con il server é da linea di comando, infatti esiste una libreria chiamata
\texttt{redis-cli}, con la quale viene semplificato notevolmente il lavoro di hacking.
In questo capitolo per fare un confronto tra comandi Redis e SQL verrà adoperata questa libreria.\\

Il linguaggio SQL é diviso in due sezioni principali:
\begin{itemize}
    \item Data Definition Language
    \item Data Manipulation Language
\end{itemize}

Per quanto riguarda il Data Definition Language, a differenza di un database relazionale, non abbiamo la possibilità di definire
domini, tabelle, vincoli sulle strutture e così via; non avremo a disposizione nessuna operazione di questo tipo.\\
Redis essendo un key-value é composto da tuple, quindi, viene creato un legame tra chiave e valore al momento dell'aggiunta di una chiave; inoltre, il tipo del valore
associato sarà definito al momento dell'aggiunta.
\section{Data Manipulation Language}
Come nel linguaggio SQL si hanno delle operazioni di query, di modifica e addirittura, anche, comandi transazionali.

\subsection{Comandi di Modifica}
come in SQL i comandi di modifica sono quelle istruzioni che permettono:
\begin{itemize}
    \item inserimento;
    \item cancellazione;
    \item modifica dei valori associati a delle chiavi.\\
\end{itemize}


Come esempio di confronto viene adoperato un database che rappresenta i profili di utenti.
Un utente é composto da: nome utente, nome, cognome, password, data di creazione.\\

In Redis ogni una tupla sarà composta nel seguente modo:
\begin{itemize}
    \item \textbf{chiave} $\to$ \textbf{utente:}nomeutente
    \item \textbf{valore} $\to$ tipo \textbf{hash}, in cui saranno contenute le informazioni riguardo a nome, cognome, password e data di creazione.
\end{itemize}
I comandi che verranno analizzati sono associati al tipo Hash, in quanto Redis dedica dei comandi specifici per ogni struttura dati.\\

Nel database relazionale si avrà un record composto da tutti gli attributi sopra definiti, la chiave primaria sarà rappresentata dallo username;\\

L'\textbf{INSERIMENTO} di un record con linguaggio SQL viene eseguito in questo modo:
\begin{lstlisting}[autogobble, style=redis-cli, language=SQL]
INSERT INTO Utenti (username, nome, cognome, password, dataCreazione)
VALUES ('matteo00', 'matteo', 'rizzo', 'mypsw', '12092020')\end{lstlisting}

Un inserimento di una tupla in Redis viene fatta nel modo seguente:\\
se utilizziamo una Hash come valore, il comando di cui abbiamo bisogno é \texttt{HSET}, il quale viene utilizzato per aggiungere una nuova tupla.
\begin{lstlisting}[autogobble, style=redis-cli, language=SQL]
>HSET utente:matteo00 nome matteo cognome rizzo password mypsw dataCreazione 12092020
(integer) 4\end{lstlisting}

Si puó notare che in Redis non abbiamo definito uno schema fisso della Hash, ma siamo noi al momento della creazione della tupla a definire le chiavi e i valori che compongono la hash,
infatti sarà possibile avere utenti con una Hash non coerente alle altre, questo ha i suoi vantaggi e svantaggi.\\

Per la \textbf{CANCELLAZIONE} di un record in SQL:
\begin{lstlisting}[autogobble, style=redis-cli, language=SQL]
DELETE FROM Utenti WHERE username = 'matteo00'\end{lstlisting}

Mentre in Redis bisogna eliminare una chiave; questo comando non dipende dal tipo di valore utilizzato, ma é un comando comune
a tutti i tipi, il comando si chiama \texttt{DEL}.

\begin{lstlisting}[autogobble, style=redis-cli, language=SQL]
>DEL utente:matteo00
(integer) 1\end{lstlisting}

In Redis la cancellazione di una tupla é poco utilizzata, solitamente viene impostata una scadenza alle chiavi, ovvero dopo un certo periodo le chiavi vengono eliminate automaticamente.
Per fare questo si utilizza il comando \texttt{EXPIRE}, in cui vengono passati come parametri la chiave ed il numero di secondi che definiscono la scadenza.
\begin{lstlisting}[autogobble, style=redis-cli, language=SQL]
>EXPIRE utente:matteo00 60
(integer) 1\end{lstlisting}
Il valore 60 indica i secondi di tempo dopo i quali la chiave \texttt{utente:matteo00} deve essere cancellata.\\
 \\


Per \textbf{MODIFICARE} i valori degli attributi di uno o piú record tramite SQL:
\begin{lstlisting}[autogobble, style=redis-cli, language=SQL]
UPDATE Utenti
SET password = 'newpsw'
WHERE username = 'matteo00'\end{lstlisting}

In Redis non vi é un comando specifico per la modifica dei valori delle tuple, ma viene utilizzato \texttt{HSET} passando come parametri del comando
i campi che vogliamo modificare con il loro nuovo valore:
\begin{lstlisting}[autogobble, style=redis-cli, language=SQL]
>HSET utente:matteo00 password newpsw
(integer) 4\end{lstlisting}
Così facendo viene modificato il valore associato alla password di \texttt{utente:matteo00}, il valore passa da \texttt{"mypsw"} a \texttt{"newpsw"}.

\subsection{Comandi di Query}
In SQL le interrogazioni hanno una struttura \texttt{select-from-where}, che puó essere estesa in diversi modi. Sono comandi
che possono anche avere una certa complessità semantica, questo é dovuto principalmente dal fatto che si possono eseguire dei join
tra diverse tabelle presenti nel database.\\
Questo non avviene in Redis, infatti le interrogazioni di una base di dati chiave-valore non possono avere una complessitá paragonabile a quella SQL.\\

Per fare una selezione di un record in SQL:
\begin{lstlisting}[autogobble,style=redis-cli, language=SQL]
SELECT *
FROM Utenti
WHERE username = 'matteo00'\end{lstlisting}

Mentre in Redis per selezionare il valore di una certa tupla, nel caso in cui sia di tipo Hash, viene messo a disposizione \texttt{HGETALL}, che ha il compito di mostrare tutto
il contenuto associato ad una certa chiave.
\begin{lstlisting}[autogobble, style=redis-cli, language=SQL]
> HGETALL utente:matteo00
1) "nome"
2) "matteo"
3) "cognome"
4) "rizzo"
5) "password"
6) "mypsw"
7) "dataCreazione"
8) "12092020"
\end{lstlisting}

Inoltre, viene messo a disposizione il comando \texttt{HGET} per ottenere solo il valore di un campo tra quelli presenti nella Hash, questo viene fatto per permettere
agli sviluppatori di risparmiare operazioni di trasformazione delle strutture all'interno del proprio software.
\begin{lstlisting}[autogobble, style=redis-cli, language=SQL]
> HGET utente:matteo00 nome
"matteo"
\end{lstlisting}

Vi é anche la possibilitá di fare una sorta di proiezione sulle chiavi utilizzando il comando \texttt{KEYS}, il quale restituisce tutte le chiavi che corrispondono al pattern passato come parametro.
\begin{lstlisting}[autogobble, style=redis-cli, language=SQL]
> KEYS utente:*
1) "utente:user1"
2) "utente:user2"
3) "utente:user3"
4) "utente:matteo00"
\end{lstlisting}
Comando che viene ampiamente utilizzato perché é possibile filtrare su chiavi che hanno un prefisso particolare.\\

Per quanto riguarda i comandi transazionali, sono giá stati definiti nel capitolo precedente nella sezione del confronto con le proprietá ACID.

\section{Store procedure}
In SQL vi é la possibilitá di definire procedure che vengono memorizzate nella base di dati come parte dello schema ed eseguite dal dbms,
ovvero con una modalitá opposta rispetto all'esecuzione dei client. Queste procedure svolgono una specifica attivitá di manipolazione dei dati.\\
Anche Redis offre un meccanismo analogo: viene consentito ai client di caricare ed eseguire script lato server.
Il vantaggio é la grande efficienza che si ottiene nella lettura e scrittura dei dati poché gli script
memorizzati sono eseguiti completamente server-side.
É garantita l'esecuzione atomica dello script e durante la sua esecuzione tutte le attivitá del server vengono bloccate.
É supportato un unico motore di scripting, ovvero l'interprete \texttt{LUA}, che é un particolare linguaggio di scripting.

Ci sono diverse modalitá lato client per invocare/caricare gli script:
\begin{itemize}
    \item Il comando \texttt{EVAL} serve per eseguire direttamente uno script; vanno forniti diversi argomenti:\\
    Il primo argomento é una stringa che consiste nel codice sorgente LUA vero e proprio;\\
    il secondo argomento indica il numero di chiavi che verranno passate nei parametri successivi (quindi é un numero);\\
    a partire dal terzo argomento si indicano i nomi delle chiavi di Redis.\\
      \\


    Un esempio di codice puó essere il seguente:
    \begin{lstlisting}[autogobble, style=redis-cli, language=SQL]
> EVAL "return {KEYS[1], KEYS[2]}" 2 key1 key2
1) "key1"
2) "key2"\end{lstlisting}
    Con questo comando eseguiamo gli script dinamicamente, perché stiamo appunto generando lo script durante il runtime dell'applicazione.
    \item É possibile caricare uno script nella cache del server Redis tramite il comando \texttt{SCRIPT LOAD} e fornendo il suo codice
    sorgente; in questo modo il server non esegue lo script, ma lo compila e lo carica nella sua cache e restituisce al client un codice, chiamato
    \textbf{SHA1 digest}, che punta direttamente allo script. Successivamente lo script viene eseguito invocando il comando \texttt{EVALSHA} e passandogli come
    primo argomento il SHA1 digest generato in precedenza e partendo dal secondo argomento si ripetono quelli di \texttt{EVAL}.

    \begin{lstlisting}[autogobble, style=redis-cli, language=SQL]
> SCRIPT LOAD "return 'example of script load!'"
"c664a3bf70bd1d45c4284ffebb65a6f2299bfc9f"  \\SHA1 digest
>EVALSHA "c664a3bf70bd1d45c4284ffebb65a6f2299bfc9f" 0
"example of script load!"\end{lstlisting}
\end{itemize}

La cache degli script Redis é sempre volatile, non é considerato come parte del database e non é persistente, quindi gli script memorizzati
sono temporanei e il contenuto della cache puó essere perso in qualsiasi momento.
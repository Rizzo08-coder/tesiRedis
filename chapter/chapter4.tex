\chapter{Interrogazioni SQL $\to$ \emph{Key-Value}}
Le capacitá piú essenziali per il lavoro su database sono saperli popolare di informazioni in modo coerente e interrogare in modo
intelligente, per ricavare l'informazione che si sta cercando in mezzo a volumi di dati che possono essere immensi.
Quindi, per raggiungere questo obiettivo adoperiamo un linguaggio per l'interrogazione dei database, il quale serve per formulare delle
query, in modo adeguato alla struttura delle informazioni.
Nei database di tipo relazionale é presente il linguaggio SQL (Structured Query Language), il quale é comune a tutti i dbms di questo tipo, quindi
é diventato negli anni un vero e proprio standard.

\paragraph{Nei NoSQL vi é un linguaggio standard da adoperare?\\}
i noSQL si caratterizzano proprio da schemi non fissi, che possono variare in modo dinamico e, soprattutto, non vi sono vincoli referenziali che possono associare
diverse tabelle all'interno del database (quindi non vi sono operazioni di join).
Per memorizzare e ricavare dati, i quali possono essere strutturati, semi-strutturati, non strutturati e polimorfici, un dbms noSQL utilizza un'ampia gamma di
tecnologie proprietarie in grado di fare questo. Quindi, é compito del produttore del dbms di dover mettere a disposizione un insieme di comandi che permettano di sfruttare pienamente le potenzialitá del NoSQL da lui fornito.
Ne deriva che il linguaggio SQL non é supportato; sono disponibili delle interfaccie di comunicazione per i principali linguaggi di programmazione che vengono
offerte direttamente al programmatore, con le quali si riesce a gestire il dialogo con il dbms.

Sulla documentazione di Redis é presente un elenco completo di tutte le API disponibili per i principali linguaggi, tramite le quali si riesce a comunicare con il server in questione.

Il modo piú rapido e semplice per interagire con il server é da linea di comando, infatti esiste una libreria chiamata
\texttt{redis-cli}, con la quale viene semplificato notevolmente il lavoro di hacking.
In questo capitolo per fare un confronto tra comandi Redis e SQL adopereremo questa libreria.\\

Sappiamo che il linguaggio SQL é diviso in due sezioni principali:
\begin{itemize}
    \item Data Definition Language
    \item Data Manipulation Language
\end{itemize}

Per quanto riguarda il Data Definition Language, a differenza di un database relazionale, non abbiamo la possibilitá di definire
domini, tabelle, vincoli sulle strutture e cosí via; non avremo a disposizione nessuna operazione di questo tipo.\\
In Redis, dovremo creare un legame tra chiave e valore, questo avviene al momento dell'aggiunta di una chiave; inoltre, il tipo del valore
associato sará definito al momento dell'aggiunta di una chiave specifica.
\section{Data Manipulation Language}
Come nel linguaggio SQL si hanno delle operazioni di query, di modifica e addirittura anche comandi transazionali.

\subsection{Comandi di Modifica}
come in SQL i comandi di modifica sono quelle istruzioni che permettono:
\begin{itemize}
    \item inserimento
    \item cancellazione
    \item modifica dei valori associati a delle chiavi
\end{itemize}


Come esempio di confronto adopero un database che rappresenta i profili di utenti,
un utente sará composto da: nomeutente, nome, cognome, password, data di creazione

Quindi in Redis avremo una chiave composta da un prefisso utente:nomeutente e il valore sará di tipo hash in cui saranno contenute le informazioni.
Mentre in un database relazionale avremo una tupla composta da tutti gli attributi sopra definiti, la chiave primaria sará rappresentata dallo username; l'intera tupla rappresenterá l'utente.

Un possibile inserimento SQL sará cosí eseguito (ovviamente dopo aver definito la struttura della tabella)
\begin{lstlisting}[autogobble]
INSERT INTO Utenti (username, nome, cognome, password, dataCreazione)
VALUES ('matteo00', 'matteo', 'rizzo', 'mypsw', '12092020')\end{lstlisting}

Analizziamo una possibile aggiunta in Redis:
Se utilizziamo una HashMap come valore il comando di cui abbiamo bisogno é \texttt{HSET}, il quale viene utilizzato per aggiungere un nuovo record.
\begin{lstlisting}[autogobble]
>HSET utente:matteo00 nome matteo cognome rizzo password mypsw dataCreazione 12092020
(integer) 4\end{lstlisting}

Si puó notare che in Redis non si ha una struttura fissa della hash, ma siamo noi al momento della creazione a definire la composizione della hash,
infatti potremo avere utenti con una hash non coerente alle altre, questo ha ovviamente i suoi vantaggi e svantaggi.\\

Per la cancellazione di un record in SQL faremo:
\begin{lstlisting}[autogobble]
DELETE FROM Utenti WHERE username = 'matteo00'\end{lstlisting}

mentre in Redis dovremo eliminare una chiave, questo comando non dipende dal tipo di valore utilizzato, ma sará un comando comune
a tutti i tipo, il comando si chiama \texttt{DEL}.

\begin{lstlisting}[autogobble]
>DEL utente:matteo00
(integer) 1\end{lstlisting}

In Redis questo approccio é poco utilizzato, solitamente viene impostata una scadenza alle chiavi, ovvero dopo un certo periodo le chiavi vengono eliminate automaticamente.
Per fare questo si utilizza il comando \texttt{EXPIRE}, in cui verrá passata la chiave a cui va impostata la scadenza e i secondi di durata.
\begin{lstlisting}[autogobble]
>EXPIRE utente:matteo00 60
(integer) 1\end{lstlisting}
Il 60 indica i secondi di tempo dopo i quali la chiave deve essere cancellata.\\

Per modificare i valori degli attributi di uno o piú record tramite SQL:
\begin{lstlisting}[autogobble]
UPDATE Utenti
SET password = 'newpsw'
WHERE username = 'matteo00'\end{lstlisting}

In Redis non vi é un comando specifico per la modifica, ma verrá utilizzato \texttt{HSET} passando come parametri del comando
i campi che vogliamo modificare con il loro nuovo valore:
\begin{lstlisting}[autogobble]
>HSET utente:matteo00 password newpsw
(integer) 4\end{lstlisting}

\subsection{Comandi di Query}
In SQL le interrogazioni hanno una struttura \texttt{select-from-where}, che puó essere estesa in diversi modi. Sono comandi
che possono anche avere una certa complessitá semantica, questo é dovuto principalmente dal fatto che si possono eseguire dei join
tra diverse tabelle presenti nel database.
Tutto questo non avviene in Redis, infatti le query di un database key-value non puó avere una complessitá paragonabile a quella SQL.

Per fare una selezione di un record in SQL:
\begin{lstlisting}[autogobble]
SELECT *
FROM Utenti
Where username = 'matteo00'\end{lstlisting}

Mentre in Redis per la struttura hash viene messo a disposizione \texttt{HGETALL}, che ha il compito di mostrare il contenuto associato ad una certa chiave.
\begin{lstlisting}[autogobble]
> HGETALL utente:matteo00
1) "nome"
2) "luca"
3) "cognome"
4) "rizzo"
5) "password"
6) "mypsw"
7) "dataCreazione"
8) "12092020"
\end{lstlisting}

Viene messo a disposizione il comando \texttt{HGET} per ottenere solo il valore di un campo tra quelli presenti nella hash, questo viene fatto per risparmiare operazioni al programmatore all'interno del proprio software.
\begin{lstlisting}[autogobble]
> HGET utente:matteo00 nome
"matteo"
\end{lstlisting}

Vi é anche la possibilitá di fare una sorta di proiezione sulle chiavi utilizzando il comando \texttt{KEYS}, il quale restituisce tutte le chiavi che corrispondono al pattern passato come parametro.
\begin{lstlisting}[autogobble]
> KEYS utente:*
1) "utente:user1"
2) "utente:user2"
3) "utente:user3"
4) "utente:matteo00"
\end{lstlisting}
Comando che viene ampiamente utilizzato perché é possibile filtrare su chiavi che hanno un prefisso particolare.\\

Per quanto riguarda i comandi transazionali, sono giá stati definiti nel capitolo precedente nella sezione del confronto con le proprietá ACID.

\section{Store procedure}



%é un vero e proprio linguaggio strutturato come sql?
